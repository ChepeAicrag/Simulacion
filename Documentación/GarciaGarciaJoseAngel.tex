\documentclass{article}
\usepackage{graphicx}
\usepackage[spanish]{babel}
\renewcommand{\baselinestretch}{1.5}

\begin{document}
	
	
	
	\begin{titlepage}
		\centering
		{\includegraphics[width=0.2\textwidth]{logo}\par}
		\vspace{1cm}
		{\bfseries\LARGE Instituo Tecnológico de Oaxaca \par}
		\vspace{1cm}
		{\scshape\Large Ingeniería en Sistemas Computacionales  \par}
		\vspace{3cm}
		{\scshape\Huge Simulación de Difusión En Redes Sociales \par}
		\vspace{3cm}
		{\itshape\Large Proyecto de Simulación \par}
		\vfill
		{\Large Autor: \par}
		{\Large José Ángel García García \par}
		\vfill
		{\Large Diciembre 2020 \par}
		\end{titlepage}
	
\tableofcontents
\newpage
	
\section{Introducción}
 
 \subsection{Planteamiento del problema}
 Las redes sociales son el medio de comunicaicón que predominan actualmente, Se han incorporado a nuestras vida hasta un nivel en el que no somos capaces de asumir que dependemos de ellas o que hacemos uso de estas, por más minimo que sea, le terminamos dando un uso. Entonces podemos decir que las redes sociales son tan influyentes en nuestras vidas que en ocasiones pareciera que nos manipularan.
  
 Hoy en día el término “red social " es muy empleado llamándose así a los diferentes sitios o páginas de internet que ofrecen registrarse a las personas y contactarse con infinidad de individuos a fin  de compartir contenidos, interactuar y crear comunidades sobre intereses similares: trabajo lecturas, amistad, juegos, relaciones amorosas, entre otros.
 
 \subsection{Defincion del sistema}
 
\section{Diseño del Modelo}
 
 \subsection{Formulación del modelo}
 \subsection{Explicación del modelo}
 
\section{Recolección de Datos}
 
 \subsection{Recolección}
 \subsection{Análisis de Datos}	
	
\section{Construcción del Modelo}

\section{Validación}

\section{Conclusiones}
	
	
	
	
	\end{document}